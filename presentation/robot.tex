\documentclass{beamer}
\usetheme{Frankfurt}
\usecolortheme{seahorse} %dove
\usepackage{remreset}
\makeatletter
\@removefromreset{subsection}{section}
\makeatother
\setcounter{subsection}{1}

%%aici poate nu am nevoie de tot
\usepackage[english]{babel}
\usepackage[T1]{fontenc}
\usepackage{amsmath}
\usepackage{amsfonts}
\usepackage{amssymb}
\usepackage[utf8x]{inputenc}
\usepackage{graphicx}
\usepackage{caption}
\usepackage{subcaption}
\usepackage[font={footnotesize}]{caption}
\usepackage{setspace}
\usepackage{float}
%\usepackage[hidelinks]{hyperref}
\usepackage{wrapfig}
\newcommand*\diff{\mathop{}\!\mathrm{d}}
\setbeamertemplate{caption}[numbered]
\usepackage[authoryear, round]{natbib}
\bibliographystyle{abbrvnat}
\renewcommand{\bibsection}{\subsubsection*{\bibname } }
%aici poate nu am nevoie de tot
\usepackage{soul}
\usepackage{animate}
\usepackage{multicol}

\setbeamertemplate{footline}[frame number]

\title[Deep Vision] %optional
{\textbf{Through the eyes of RoboFace}}
 
\subtitle{Deep Vision -- final project}
 
\author{Letiția Elena Pârcălăbescu}
 
\date{8\textsuperscript{th} of February 2017}

\begin{document}
\frame{\titlepage}

%\begin{frame}
%\frametitle{Structure}
%\tableofcontents
%\end{frame}

\section{Problem}
\begin{frame}{RoboFace}{Problem description}
	\begin{figure}
		\centering
		\vspace*{-18mm}
		\includegraphics[width=0.8\linewidth,angle=270]{figures/roboFace}
		\label{fig:face}
	\end{figure}
\end{frame}

\begin{frame}{RoboFace}{Problem description}
		\vspace*{-11.6mm}
		\hspace*{2.85cm}
		\includegraphics[width=0.61\linewidth]{figures/webcam}
\end{frame}

\section{Dataset}
\begin{frame}{CelebA Dataset \cite{website}}
\centering
	202.599 images,
	10.177 identities,
	40 attributes per image 
	\begin{figure}
		\centering
		\includegraphics[width=0.9\linewidth]{figures/CelebA}
		\label{fig:celebA}
	\end{figure}
\end{frame}

\begin{frame}{CelebA Dataset}
	\begin{figure}
		\centering
		\includegraphics[width=0.9\linewidth]{figures/CelebAExamples}
		\label{fig:celebAExamples}
	\end{figure}
\end{frame}

\begin{frame}{CelebA dataset}{Attribute selection}
	chose 13 labels out of 40 available
	\begin{multicols}{2}
	\begin{enumerate}
	\item Black Hair
	\item Blond Hair
	\item Brown Hair
	\item Eyeglasses
	\item Gray Hair
	\item Male
	\item Mouth Slightly Open
	\item No Beard
	\item Smiling
	\item Straight Hair
	\item Wavy Hair
	\item Wearing Earrings
	\item Wearing Lipstick
	\end{enumerate}
	\end{multicols}
\end{frame}

\section{Paper}
% http://www.cv-foundation.org/openaccess/content_iccv_2015/papers/Liu_Deep_Learning_Face_ICCV_2015_paper.pdf
\begin{frame}{Deep Learning Face Attributes in the Wild}{\cite{liu2015faceattributes}}
	\centering
	Joint face localisation and attribute prediction using only image-level attribute tags.
	\begin{figure}
			\centering
			\includegraphics[width=\linewidth]{figures/paperArchitecture}
			\label{fig:paperarchi}
	\end{figure}
\end{frame}

\begin{frame}{Deep Learning Face Attributes in the Wild}{\cite{liu2015faceattributes}}
	\centering
	Response map of the LNet (on attributes).
	\begin{figure}
			\centering
			\includegraphics[width=\linewidth]{figures/response}
			\label{fig:response}
	\end{figure}
\end{frame}

\section{Own Approach}
\begin{frame}{Own Approach}{Architecture}
	\begin{enumerate}
			\item 	32 of neurons in each convolutional layer
			\item Activations: Relu
			\item 9 x 9 convolution $\rightarrow$ Max Pooling
			\item	7 x 7 convolution $\rightarrow$  Max Pooling
			\item	5 x 5 convolution $\rightarrow$  Max Pooling
			\item	3 x 3 convolution $\rightarrow$  Max Pooling
			\item	3 x 3 convolution $\rightarrow$  Max Pooling
			\item	Dropout(0.25)
			\item	512 Dense
			\item	Dropout(0.5)
			\item	13 Dense
			\item	Sigmoid
			\item	Binary Crossentropy
			\item	Adadelta optimiser
			\item Overall: 125,709 parameters
	\end{enumerate}
\end{frame}

\begin{frame}{Own Approach}{How to not do it!}
Without data normalisation: beautiful training curve, BUT...
\hspace*{-2.4cm}
	\includegraphics[height=0.6\textheight]{figures/lossBad} 
\end{frame}

\begin{frame}{Own Approach}{How to not do it!}
... BUT no generalisation capability at all! \\
The predicted classes on the real world images from the robot are: \\
\centering
Male (0.89), No beard (0.76) \\
\includegraphics[height=0.6\textheight]{figures/failMaleNoBeard} 
\end{frame}

\begin{frame}{Data normalisation}{How to really do it!}
\begin{enumerate}
\item center the eyes
\item resize images to have the same inter ocular distance
\item rotate image to make the inter ocular line look horizontal
\item resize image to 128 x 128 pixels
\item subtract the mean face
\end{enumerate}
\centering
	\includegraphics[height=0.6\textheight]{figures/meanFace} 
\end{frame}

\begin{frame}{Own Approach}{Training curves}
\hspace*{-2.4cm}
	\includegraphics[height=0.6\textheight]{figures/lossGood} 
\end{frame}

\begin{frame}{Own Approach}{Final accuracy per class}
\centering
overall accuracy of 90\%
	\includegraphics[width=1.1\textwidth]{figures/accuracy_on_classes} 
\end{frame}

\begin{frame}{Exemplary results}
\centering
'Black Hair', 'Mouth Slightly Open', 'No Beard', 'Smiling', 'Wearing Earrings', 'Wearing Lipstick' \\
\includegraphics[height=0.3\textheight]{figures/leti_normalised110} \\ 
'Black Hair', 'Mouth Slightly Open', 'No Beard', 'Straight Hair', 'Wearing Lipstick' \\
\includegraphics[height=0.3\textheight]{figures/leti_normalised111} 
\end{frame}

\begin{frame}{Exemplary results}
'Black Hair', 'Mouth Slightly Open', 'No Beard', 'Smiling', 'Straight Hair', 'Wearing Earrings', \textcolor{red}{'Wearing Lipstick}
 \\
 \centering
\includegraphics[height=0.3\textheight]{figures/leti_normalised24}\\
'Black Hair', \textcolor{red}{'Male'}, 'Straight Hair' -- only 2 out of 45 examples. \\
\includegraphics[height=0.3\textheight]{figures/leti_normalised0}
\end{frame}



\section{References}
\begin{frame}
	\frametitle{References}
	\footnotesize
	\bibliography{mybib.bib}
\end{frame}

\end{document}